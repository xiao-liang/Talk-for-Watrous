\documentclass[aspectratio=1610, 12pt, xcolor={dvipsnames}]{beamer}
\usefonttheme{professionalfonts}
\usepackage{siunitx}
\usepackage{newtxtext}
\usepackage{bm}
% Possible aspect ration: 1610, 149, 54, 43 and 32.
\usepackage{Xiao-slides}

% Comment Macro
\newcommand{\xiao}[1]{{\color{Red}XIAO: #1}}
\newcommand{\gray}[1]{{\color{Gray} #1}}
\newcommand{\green}[1]{{\color{ForestGreen} #1}}
\newcommand{\beamerblue}[1]{{\color{BeamerBlue}#1}\xspace}

% Macros:
% Case-insensitive, Alphabetical ordering
\newcommand{\Adv}{\ensuremath{\mathcal{A}}\xspace}

\newcommand{\Aux}{\ensuremath{\mathsf{Aux}}\xspace}
\newcommand{\Bad}{\ensuremath{\mathsf{Bad}}\xspace}
\newcommand{\BBCom}{\ensuremath{\mathsf{BBCom}}\xspace}
\newcommand{\BBOWFP}{\ensuremath{\mathsf{BB}\text{-}\mathsf{OWFP}}\xspace}
\newcommand{\BBProve}{\ensuremath{\mathsf{BBProve}}\xspace}
\newcommand{\Bdv}{\ensuremath{\mathcal{B}}\xspace}
\newcommand{\bits}{\ensuremath{\{0,1\}}\xspace}
\newcommand{\blue}[1]{{\color{NavyBlue} #1}}
\newcommand{\BPP}{\ensuremath{\mathcal{BPP}}\xspace}
\newcommand{\ch}{\ensuremath{\mathsf{ch}}\xspace}
\newcommand{\cind}{\ensuremath{\stackrel{\text{c}}{\approx}}\xspace}
\newcommand{\Circuit}{\ensuremath{\mathsf{Cir}}\xspace}
\newcommand{\CKTVAL}{\ensuremath{\textsc{CktVal}}\xspace}
\newcommand{\cm}{\ensuremath{\mathsf{cm}}\xspace}
\newcommand{\com}{\ensuremath{\mathsf{com}}\xspace}
\newcommand{\Com}{\ensuremath{\mathsf{Com}}\xspace}
\newcommand{\ConPar}{\ensuremath{n}}
\newcommand{\CRHF}{\ensuremath{\mathsf{CRHF}}\xspace}
\newcommand{\cZK}{\ensuremath{c\mathcal{ZK}}\xspace}
\newcommand{\Dec}{\ensuremath{\mathsf{Dec}}\xspace}
\newcommand{\Dist}{\ensuremath{\mathcal{D}}\xspace}

\newcommand{\Domain}{\ensuremath{\mathsf{D}}\xspace}
\newcommand{\Dstng}{\ensuremath{\mathcal{D}}\xspace}

\newcommand{\Easy}{\ensuremath{\mathsf{Easy}}\xspace}
\newcommand{\ECC}{\ensuremath{\mathsf{ECC}}\xspace}
\newcommand{\Enc}{\ensuremath{\mathsf{Enc}}\xspace}
\renewcommand{\epsilon}{\ensuremath{\varepsilon}\xspace}

\newcommand{\Error}{\ensuremath{\delta}\xspace}
\newcommand{\Eval}{\ensuremath{\mathsf{Eval}}\xspace}
\newcommand{\egval}{\ensuremath{\mathsf{\lambda}}\xspace}

\newcommand{\Event}{\ensuremath{\mathsf{Event}}\xspace}
\newcommand{\Exec}{\ensuremath{\mathsf{Exec}}\xspace}
\newcommand{\ExtCom}{\ensuremath{\mathsf{ExtCom}}\xspace}
\newcommand{\Ext}{\ensuremath{\mathsf{Ext}}\xspace}

\newcommand{\Field}{\ensuremath{\mathbb{F}}\xspace}
\newcommand{\Func}{\ensuremath{\mathcal{F}}\xspace}

\newcommand{\given}[1][\big]{\:#1\vert\:}
\newcommand{\Good}{\ensuremath{\mathsf{Good}}\xspace}


\newcommand{\Ham}{\ensuremath{\Delta}\xspace}
\newcommand{\Hard}{\ensuremath{\mathsf{Hard}}\xspace}

\newcommand{\idind}{\ensuremath{\stackrel{\text{i.d.}}{=\mathrel{\mkern-3mu}=}}\xspace}
\newcommand{\idm}{\ensuremath{\mathbbm{1}}\xspace}
\newcommand{\IDEAL}{\ensuremath{\mathsf{IDEAL}}\xspace}
\newcommand{\Iff}{\ensuremath{\Leftrightarrow}\xspace}
\newcommand{\Ind}{\ensuremath{\mathsf{Ind}}\xspace}
\newcommand{\Input}{\ensuremath{\mathsf{in}}\xspace}
\newcommand{\IP}{\ensuremath{\mathcal{IP}}\xspace}

\newcommand{\KGen}{\ensuremath{\mathsf{KGen}}\xspace}

\newcommand{\Lang}{\ensuremath{\mathcal{L}}\xspace}

\newcommand{\NCKTVAL}{\ensuremath{\textsc{NdCktVal}}\xspace}
\newcommand{\negl}{\ensuremath{\mathsf{negl}}\xspace}
\newcommand{\NoHard}{\ensuremath{\mathsf{NoHard}}\xspace}
\newcommand{\NP}{\ensuremath{\mathsf{NP}}\xspace}
\newcommand{\np}{\NP}

\newcommand{\om}[1]{{\noindent \color{magenta}OM: #1}}
\newcommand{\Ora}{\ensuremath{\mathcal{O}}\xspace}
\newcommand{\ora}{\ensuremath{\mathsf{O}}\xspace}
\newcommand{\Output}{\ensuremath{\mathsf{Out}}\xspace}
\newcommand{\OWFP}{\ensuremath{\mathsf{OWFP}}\xspace}
\newcommand{\OWF}{\ensuremath{\mathsf{OWF}}\xspace}
\newcommand{\owf}{\OWF}

\renewcommand{\P}{\ensuremath{\mathcal{P}}\xspace}
\newcommand{\Path}{\ensuremath{\mathsf{P}}\xspace}
\newcommand{\PBCom}{\ensuremath{\mathsf{PBCom}}\xspace}
\newcommand{\PHCom}{\ensuremath{\mathsf{PHCom}}\xspace}
% \newcommand{\pick}{\ensuremath{\xleftarrow{\$}}\xspace}
\newcommand{\pick}{\ensuremath{\leftarrow}\xspace}
\newcommand{\poly}{\ensuremath{\mathsf{poly}}\xspace}
\newcommand{\PPT}{\ensuremath{\mathrm{PPT}}\xspace}
\newcommand{\PRF}{\ensuremath{\mathsf{PRF}}\xspace}
\newcommand{\PRG}{\ensuremath{\mathsf{PRG}}\xspace}
\newcommand{\Prob}[2][]{\ensuremath{\Pr#1[ \text{#2} #1]}}
\newcommand{\Prot}{\ensuremath{\mathrm{\Pi}}\xspace}

\newcommand{\Range}{\ensuremath{\mathsf{R}}\xspace}
\newcommand{\reg}[1]{\ensuremath{\mathsf{\uppercase{#1}}}\xspace}

\newcommand{\REAL}{\ensuremath{\mathsf{REAL}}\xspace}
\newcommand{\Recon}{\ensuremath{\mathsf{Recon}}\xspace}
\newcommand{\red}[1]{{\color{red} #1}}
\newcommand{\Relation}{\ensuremath{\mathcal{R}}\xspace}
\newcommand{\RouPar}{\ensuremath{k}}

\newcommand{\SBCom}{\ensuremath{\mathsf{SBCom}}\xspace}
\newcommand{\SecPar}{\ensuremath{n}\xspace}
\newcommand{\Seq}[1]{\ensuremath{\{#1\}}\xspace}
\newcommand{\Set}[1]{\ensuremath{\{#1\}}\xspace}
\newcommand{\Share}{\ensuremath{\mathsf{Share}}\xspace}
\newcommand{\SHCom}{\ensuremath{\mathsf{SHCom}}\xspace}
\newcommand{\Sim}{\ensuremath{\mathsf{Sim}}\xspace}
\newcommand{\sind}{\ensuremath{\stackrel{\text{s}}{\approx}}\xspace}
\newcommand{\ST}{\ensuremath{\mathsf{ST}}\xspace}

\newcommand{\Trans}{\ensuremath{\tau}\xspace}
\newcommand{\tensor}{\ensuremath{\otimes}\xspace}



\newcommand{\Valid}{\ensuremath{\mathsf{Valid}}\xspace}
\newcommand{\Vectbf}[1]{\ensuremath{\mathbf{#1}}\xspace}
\newcommand{\Vect}[1]{\ensuremath{\overline{#1}}\xspace}
\newcommand{\View}{\ensuremath{\mathsf{View}}\xspace}
\newcommand{\VSSCom}{\ensuremath{\mathsf{VSSCom}}\xspace}
\newcommand{\VSS}{\ensuremath{\mathsf{VSS}}\xspace}


\newcommand{\xor}{\ensuremath{\oplus}\xspace}

\newcommand{\ZK}{\ensuremath{\mathsf{ZK}}\xspace}
\newcommand{\zk}{\ZK}
\newcommand{\ZKCnP}{\ensuremath{\mathsf{ZKCnP}}\xspace}
\newcommand{\ZKP}{\ensuremath{\mathsf{ZKP}}\xspace}
\newcommand{\zkp}{\ZKP}


\begin{document}
 

% \frame{\titlepage}
{ % these braces make the change local to the single frame
   \setbeamertemplate{footline}{\centerline{\scriptsize Git repo for these slides and {\LaTeX} source code: \href{https://github.com/xiao-liang/Talk-for-Watrous}{https://github.com/xiao-liang/Talk-for-Watrous}}} % add a footnote for this frame only
\begin{frame}
\vspace{4em}
\centerline{\Large\beamerblue{The Watrous Post-Quantum Zero-Knowledge Proof}}
\vspace{3em}
\centerline{\beamerblue{A Crypto Reading Group Talk}}
\vspace{1em}
\centerline{by}
\vspace{1em}
\centerline{Xiao Liang}
\vspace{2em}
\centerline{\small\textsc{Stony Brook University}}
\centerline{\small and}
\centerline{\small\textsc{Max-Planck Institute (Security and Privacy)}}
\vspace{1em}
\centerline{\small Aug.\ 2nd, 2021}
\end{frame}
}

\begin{frame}
\frametitle{Post-Quantum ZK for NP}

The model:
\begin{itemize}
\item
Classical $P$ and $V$
\item
ZK system for \NP languages
\item
$V^*$ can be quantum. 
\begin{itemize}
\item
Modeled as a quantum polynomial-time (QPT) Turing machine.
\item
equivalently (and more preferred in quantum-computing literature), poly-size quantum circuits. 
\item 
Non-uniformity: $V^*$ has an auxiliary quantum state that depends only on the security para.\ $\SecPar$. More accurately,
$$V^* = \Set{\mathsf{QC}_\SecPar, \ket{\psi_\SecPar}}_{\SecPar\in \mathbb{N}}$$
\end{itemize}
\end{itemize}
\end{frame}


\begin{frame}
\frametitle{Post-Quantum (Black-Box) ZK Is Hard}

Why's {\bf rewinding} hard?
\begin{itemize}
\item
    information gain VS state disturbance
\item
 the no-cloning theorem
\end{itemize}
~\\
The major result in \cite{DBLP:conf/stoc/Watrous06}: a quantum rewinding lemma 
\end{frame}


\begin{frame}
\frametitle{Some Historical Notes}
Techniques inspired by Marriot-Watrous \cite{DBLP:conf/coco/MarriottW04} 
\begin{itemize}
\item error-gap amplification for QMA using only 1 witness state
\end{itemize}
~\\
First published at STOC'06 \cite{DBLP:conf/stoc/Watrous06}
\begin{itemize}
\item
Explicit connection to \cite{DBLP:conf/coco/MarriottW04}
\item
Simple, ad hoc proof
\item
This talk mainly focuses on this version
\item
The notation herein is consistent with this version
\end{itemize}
~\\
Then, on SIAM Journal of Computing in 2009 \cite{DBLP:journals/siamcomp/Watrous09}
\begin{itemize}
\item
Abstracts out a general quantum rewinding lemma
\item
Hides the connection with Marriot-Watrous
\item
We'll also see the high-level idea of this version 
\end{itemize}
\end{frame}


\begin{frame}
\frametitle{Agenda for Today}
\begin{itemize}
\item
Prove quantum ZK for the Graph Isomorphism protocol \cite{DBLP:conf/focs/GoldreichMW86} \blue{(in detail)}
\begin{itemize}
\item 
Originally ad hoc \cite{DBLP:conf/stoc/Watrous06}
\item
We'll take a general perspective 
\end{itemize}
\item
Extends to the Graph-3-coloring Protocol \cite{DBLP:conf/focs/GoldreichMW86} in the ideal \Com model \blue{(simple)}
\begin{itemize}
\item
General quantum rewinding lemma
\end{itemize}
\item
G3C ZK with computationally-secure $\Com$ \blue{(simple-yet-tedious)}
\begin{itemize}
\item
Rewinding lemma in its most general form --- allowing small perturbations
\item
the widely-used version in crypto literature
\end{itemize}
\end{itemize}
\end{frame}

\begin{frame}
        \frametitle{GMW ZK for Graph Isomorphism (GI)}

Some Remarks:
\begin{itemize}
\item
GI is not known to be \NP-complete.
\item
the 1st message of the GMW GI protocol is perfectly uniform.
\end{itemize}
~\\

{\bf Input for $P$:} statement $(G_0, G_1) \in \mathcal{G}_\SecPar \times\mathcal{G}_\SecPar$, witness $w = \sigma$ s.t.\ $\sigma(G_1) = G_0$

{\bf Input for $V$:}  $(G_0, G_1)$
\begin{enumerate}
\item 
$P$ samples $\pi \pick S_\SecPar$, sends $H = \pi(G_0)$
\item
$V$ sends $a \pick \bits$
\item
$P$ sends $\tau = \pi \circ \sigma^a$
\end{enumerate}
{\bf $V$'s decision:} accept iff $\tau(G_a) = H$
\\~

{\bf Classical Sim:} guess the bit $b$. Set $H = \pi(G_b)$. Win if $b == a$.

\end{frame}



\begin{frame}
        \frametitle{Modeling in Quantum Way}


{\bf Model a Quantum $V^*$:} circuit family $\Set{\vb{V}_H}_{ H \in \mathcal{G}_\SecPar}$, auxiliary input $\ket{\psi}$
\begin{itemize}
	\item Receives $H$ from P
	\item
	Perform $\vb{V}_H \ket{\psi}_\reg{w} \ket{0}_\reg{v} \ket{0}_\reg{a} = \alpha_0 \ket{\psi_0}_\reg{wv} \ket{0}_\reg{a} + \alpha_1 \ket{\psi_1}_\reg{wv} \ket{1}_\reg{a}$ 
	\begin{itemize}
	\item $\reg{v}$: work space
	\item $\reg{a}$: single-qubit register to store $V^*$'s challenge.
	\item Note that $\vb{V}_H$ operates on space $\reg{w}\tensor \reg{v} \tensor \reg{a}$
	\end{itemize}
	
\end{itemize}
\end{frame}

\begin{frame}
\frametitle{Modeling in Quantum Way}
View the protocol through a quantum lens: 
\begin{itemize}
	\item
	The full space $\reg{w} \tensor \reg{x}$, where $\reg{x} = \reg{v} \tensor \reg{a} \tensor \reg{y} \tensor \reg{b} \tensor \reg{z}$
\item
Sim performs \blue{(classical Sim in superposition)} $$\vb{T} \ket{0}_\reg{YBZ} = \frac{1}{\sqrt{2\SecPar!}} \sum_{b\in \bits} \sum_{\pi \in S_\SecPar} \ket{\pi(G_b)}_\reg{y} \ket{b}_\reg{B} \ket{\pi}_\reg{z}$$
\item
$V$ apply $\vb{V} = \sum_{H \in \mathcal{G}} \vb{V}_H \tensor \ketbra{H}_\reg{y} \tensor \idm_\reg{bz}$ on the full space $\reg{w}\tensor\reg{x}$
\begin{itemize}
	\item recall that $\vb{V}_H$ operates on $\ket{\psi}_\reg{w} \ket{0}_\reg{v} \ket{0}_\reg{a}$
\item 
corresponding to the exec.\ in super-position
\item
Output format:
$$ \alpha_{00}\ket{\psi_{00}}\ket{00}_\reg{ab} +  \alpha_{01}\ket{\psi_{01}}\ket{01}_\reg{ab} + \alpha_{10}\ket{\psi_{10}}\ket{10}_\reg{ab} +\alpha_{11}\ket{\psi_{11}}\ket{11}_\reg{ab}$$
\end{itemize}
\end{itemize}
In summary, the protocol up to step 2 is:
\begin{equation} \label{eq:half-Q}
\underbrace{\vb{V}\vb{T}}_{\text{on}~\reg{w}\tensor \reg{x}} (\ket{\psi}_\reg{w} \ket{0}_{\reg{x}= \reg{VAYBZ}}) ~~\Iff~~ \underbrace{\vb{V}\vb{T}(\idm_\reg{w} \tensor \ket{0}_\reg{x})}_{\text{only on}~\reg{w}} \ket{\psi}
\end{equation}
\end{frame}

\begin{frame}
\frametitle{Measuring the Guess}

Define a binary-outcome measurement on the full space $\reg{w}\tensor\reg{x}$: 
\begin{itemize}
\item $\vb{\Pi}_0 = \ketbra{00}_\reg{ab} + \ketbra{11}_\reg{ab}, ~~\vb{\Pi}_1 \coloneqq \idm_\reg{ab} - \vb{\Pi}_0$
\item 
work on the full space $\reg{w}\tensor\reg{x}$. Just tensor identities on registers other than $\reg{ab}$
\end{itemize}
~\\


Performing $\Set{\vb{\Pi}_0, \vb{\Pi}_1}$ on $\vb{V}\vb{T}\ket{\psi}_\reg{w}\ket{0}_\reg{x}$:
\begin{itemize}
\item w.p.\ 
$\Tr\big(\bra{\psi}\vb{Q} \ket{\psi}\big)$, the outcome is $0$.
\item w.p.\  
$\Tr\big(\bra{\psi}( \idm_\reg{w}- \vb{Q} )\ket{\psi}\big)$, the outcome is $1$.
\end{itemize}
where $\vb{Q}	 = (\idm_\reg{w} \tensor \bra{0}_\reg{x}) \vb{T}^\dagger \vb{V}^\dagger \vb{\Pi}_0 \vb{T} \vb{V} (\idm_\reg{w} \tensor \ket{0}_\reg{x})$. (See Expression \eqref{eq:half-Q}.)
\\~

Two important facts:
\begin{itemize}
\item $\Set{\vb{Q}, \idm_\reg{w}- \vb{Q}}$ form a POVM
\item
$\Tr\big(\bra{\psi}\vb{Q} \ket{\psi}\big) = \Tr\big(\bra{\psi}( \idm_\reg{w}- \vb{Q} )\ket{\psi}\big) = \frac{1}{2}$, \blue{independent of $\ket{\psi}$}. (Cuz 1st msg.\ of GI prot.\ is perfectly uniform.)
\end{itemize}
$\Rightarrow ~~ \vb{Q}  = \idm_\reg{w}- \vb{Q} = \frac{1}{2}\idm_\reg{w}$
\end{frame}



\begin{frame}
\frametitle{An Important Lemma}
Let $\vb{\Delta}_0 \coloneqq \idm_\reg{w} \tensor \ketbra{0}_\reg{x}$.
\begin{itemize}
\item 
$\vb{\Delta}_0$ projects register $\reg{x}$ to all-$0$ qubits. 
\item
$\vb{\Delta}_0 = \vb{\Delta}^\dagger_0$ 
\item
$\vb{\Delta}_1 \coloneqq \idm_\reg{wx}  -  \vb{\Delta}_0$. The $\Set{\vb{\Delta}_0, \vb{\Delta}_1}$ form a POVM.
\end{itemize}
{\small
\begin{LemmaBox}[label={lem:technical}]{}
 For all $\ket{\psi}\in \mathcal{H}(\reg{w})$, $\ket{\gamma_0} = \ket{\psi}_\reg{w}\ket{0}_\reg{x}$ is an eigenvector of $\underbrace{\vb{\Delta}^\dagger_0 \vb{T}^\dagger \vb{V}^\dagger \vb{\Pi}_0 \vb{V} \vb{T} \vb{\Delta}_0}_{\coloneqq \vb{M}}$ with corresponding eigenvalue $\lambda = 1/2$.
\end{LemmaBox}
}
{\bf Proof.} Recall $\vb{Q}	 = (\idm_\reg{w} \tensor \bra{0}_\reg{x}) \vb{T}^\dagger \vb{V}^\dagger \vb{\Pi}_0 \vb{V} \vb{T} (\idm_\reg{w} \tensor \ket{0}_\reg{x}) = \frac{1}{2}\idm_\reg{w}$.
\begin{align*}
\Rightarrow ~~ & 
\vb{\Delta}^\dagger_0 \vb{T}^\dagger \vb{V}^\dagger \vb{\Pi}_0  \vb{V} \vb{T} \vb{\Delta}_0 = (\idm_\reg{w}\tensor\ket{0}_\reg{x}) \vb{Q} (\idm_\reg{w}\tensor\bra{0}_\reg{x}) = \frac{1}{2} \idm_\reg{w} \tensor \ketbra{0}_\reg{x} \\
\Rightarrow ~~ &  \forall \ket{\psi}, \vb{\Delta}^\dagger_0 \vb{T}^\dagger \vb{V}^\dagger \vb{\Pi}_0  \vb{V} \vb{T} \vb{\Delta}_0 \underbrace{\ket{\psi}_\reg{w} \ket{0}_\reg{x}}_{\ket{\gamma_0}} = \big(\frac{1}{2} \idm_\reg{w} \tensor \ketbra{0}_\reg{x}\big) \underbrace{\ket{\psi}_\reg{w} \ket{0}_\reg{x}}_{\ket{\gamma_0}} = \frac{1}{2}\underbrace{\ket{\psi}_\reg{w} \ket{0}_\reg{x}}_{\ket{\gamma_0}}
\end{align*}
\end{frame}



\begin{frame}
\frametitle{Marriot-Watrous Lemma}
{\fontsize{9pt}{0pt}\selectfont
\begin{LemmaBox}[label={lem:marriot-watrous}]{Marriot-Watrous \cite{DBLP:conf/coco/MarriottW04}}
Given unitary $\vb{U}$, proj.\ mnt.\ $\Set{\vb{\Pi}_0, \vb{\Pi}_1}$ and $\Set{\vb{\Delta}_0, \vb{\Delta}_1}$. 
Assume $\ket{\gamma_0}$ is an evec.\ of $\vb{\Delta}_0\vb{U}^\dagger\vb{\Pi}_0\vb{U}\vb{\Delta}_0$ with eval.\ $\egval$. Define 
$$\ket{\delta_0} \coloneqq \frac{\vb{\Pi}_0\vb{U}\ket{\gamma_0}}{\sqrt{\egval}}, ~~ \ket{\delta_1} \coloneqq \frac{\vb{\Pi}_0\vb{U}\ket{\gamma_0}}{\sqrt{1-\egval}},~~ \ket{\gamma_1} \coloneqq \frac{\vb{\Delta}_1\vb{U}^\dagger\ket{\delta_0}}{\sqrt{1-\egval}}.$$
Then, $\braket{\gamma_0}{\gamma_1} = \braket{\delta_0}{\delta_1} = 0$ and 
\begin{align*}
\vb{U}\ket{\gamma_0} = \sqrt{\egval}\ket{\delta_0} + \sqrt{1 - \egval}\ket{\delta_1}~~~~~~~~~~~~ & \vb{U}^\dagger\ket{\delta_0} = \sqrt{\egval}\ket{\gamma_0} + \sqrt{1-\egval}\ket{\gamma_1} \\
\vb{U}\ket{\gamma_1} = \sqrt{1-\egval}\ket{\delta_0} - \sqrt{\egval}\ket{\delta_1} ~~~~~~~~~~~~ & \vb{U}^\dagger\ket{\delta_1} = \sqrt{1-\egval}\ket{\gamma_0} - \sqrt{\egval}\ket{\gamma_1} 
\end{align*}

\end{LemmaBox}}
\blue{(draw the evolution diagram)}
$$
\mqty{
\ket{\gamma_0} & ~~~~~~~~~~~~~~~ & \green{\ket{\delta_0}} & ~~~~~~~~~~~~~~~ & \ket{\gamma_0} & ~~~~~~~~~~~~~~~ & \green{\ket{\delta_0}} & ~~~~~~~~~~~~~~~ & \ket{\gamma_0} &\cdots\\ 
~ & ~ & ~ & ~ & ~ \\
~ & ~ & ~ & ~ & ~\\
~ & ~ & ~ & ~ & ~\\
~ & ~~~~~~~~~~~~~~~ & \ket{\delta_1} & ~~~~~~~~~~~~~~~ & \red{\ket{\gamma_1}} & ~~~~~~~~~~~~~~~ & \ket{\delta_1} &  ~~~~~~~~~~~~~~~ & \red{\ket{\gamma_0}} &\cdots
} 
$$
\end{frame}


\begin{frame}
\frametitle{In Our Setting: Marriot-Watrous $+$ Post-Mnt.\ Selection}
In our setting, we have $\vb{U} = \vb{V}\vb{T}$, $\lambda = 1/2$, and $\ket{\gamma_0} = \ket{\psi}_\reg{w} \ket{0}_\reg{x}$\\[0.5em]

Lemma \ref{lem:marriot-watrous}~ $\Rightarrow ~ \ket{\gamma_0} = \frac{1}{\sqrt{2}}\ket{\delta_0} + \frac{1}{\sqrt{2}}\ket{\delta_1}$, and the following:
{\fontsize{10pt}{0pt}\selectfont$$
\ket{\delta_0} = \sqrt{2}\vb{\Pi}_0 \vb{V} \vb{T}\ket{\gamma_0}, ~~~~\vb{T}^\dagger \vb{V}^\dagger \ket{\delta_1} = \frac{1}{\sqrt{2}}\ket{\gamma_0} - \frac{1}{\sqrt{2}}\ket{\gamma_1}, ~~~~\vb{VT} (\frac{1}{\sqrt{2}}\ket{\gamma_0} + \frac{1}{\sqrt{2}}\ket{\gamma_1}) = \ket{\delta_0}
$$}

% $~~\ket{\delta_1} = \sqrt{2}\vb{\Pi}_1 \vb{V} \vb{T}\ket{\gamma_0}, ~~ \ket{\gamma_1} = \sqrt{2}\vb{\Delta}_1 \vb{T}^\dagger \vb{V}^\dagger\ket{\gamma_0}$
Starting with $\ket{\gamma_0} \rightarrow \vb{VT} \ket{\gamma_0} \rightarrow$ measurement $\Set{\vb{\Pi}_0, \vb{\Pi}_1}$:
\begin{itemize}
\item
w.p.\ $1/2$, it is $\green{\ket{\delta_0}}$ --- we are done!
\item
w.p.\ $1/2$, it is $\ket{\delta_1}$
\begin{itemize}
\item Key observation: $\vb{T}^\dagger \vb{V}^\dagger \ket{\delta_1} = \frac{1}{\sqrt{2}}\ket{\gamma_0} - \frac{1}{\sqrt{2}}\ket{\gamma_1}$
\item \red{If we can flip the phase of the 2nd term} $\Rightarrow ~~\frac{1}{\sqrt{2}}\ket{\gamma_0} + \frac{1}{\sqrt{2}}\ket{\gamma_1}$.
\item
Then, simply do $\vb{VT}(\frac{1}{\sqrt{2}}\ket{\gamma_0} + \frac{1}{\sqrt{2}}\ket{\gamma_1}) = \green{\ket{\delta_0}}$
\end{itemize}
\end{itemize}
\red{Yes, we can!} (next slide)
\end{frame}

\begin{frame}
\frametitle{Phase Flip for the 2nd Term}

We want: $\frac{1}{\sqrt{2}}\ket{\gamma_0} - \frac{1}{\sqrt{2}}\ket{\gamma_1} ~~~\rightarrow~~~ \frac{1}{\sqrt{2}}\ket{\gamma_0} + \frac{1}{\sqrt{2}}\ket{\gamma_1}$ \\[0.5em]

Recall the following 
\begin{itemize}
\item
$\ket{\gamma_0} = \ket{\psi}_\reg{w} \ket{0}_\reg{x}$ and $\vb{\Delta}_0 = \idm_\reg{w}\tensor\ketbra{0}_\reg{x}$
\item
$\Rightarrow ~~ \vb{\Delta}_0 \ket{\gamma_0} = \ket{\gamma_0}$
\item
Lemma \ref{lem:marriot-watrous} says $\ket{\gamma_1} = \sqrt{2}\vb{\Delta}_1 \vb{T}^\dagger \vb{V}^\dagger \ket{\delta_0} ~~\Rightarrow ~~ \vb{\Delta}_0 \ket{\gamma_1} = 0$
\end{itemize}

Therefore, it is not hard to come up with the following idea:
\begin{align*}
\underbrace{(2\vb{\Delta}_0 - \idm_\reg{wx})}_{=\vb{\Delta}_0 - \vb{\Delta}_1} (\frac{1}{\sqrt{2}}\ket{\gamma_0} - \frac{1}{\sqrt{2}}\ket{\gamma_1})
&= \frac{2}{\sqrt{2}}\vb{\Delta}_0 \ket{\gamma_0} - \frac{2}{\sqrt{2}}\vb{\Delta}_0 \ket{\gamma_1} - \frac{1}{\sqrt{2}}\ket{\gamma_0} + \frac{1}{\sqrt{2}}\ket{\gamma_1} \\
&= \frac{2}{\sqrt{2}}\ket{\gamma_0} - 0 - \frac{1}{\sqrt{2}}\ket{\gamma_0} + \frac{1}{\sqrt{2}}\ket{\gamma_1} \\
&= \frac{1}{\sqrt{2}}\ket{\gamma_0} + \frac{1}{\sqrt{2}}\ket{\gamma_1}
\end{align*}
\end{frame}





\begin{frame}
\frametitle{Summarizing the Watrous Simulator}
\begin{itemize}
\item
Start with $\ket{\gamma_0}_\reg{xw}=\ket{\psi}_\reg{x} \ket{0}_\reg{w}$
\item
Perform $\vb{VT}\ket{\gamma_0}_\reg{xw}$
\item
Perform measurement $\Set{\vb{\Pi}_0, \vb{\Pi}_1}$
\begin{itemize}
\item If outcome is 0 --- guessed correctly (in $\green{\ket{\delta_0}}$). Go next step.
\item Otherwise, we are in $\ket{\delta_1} = \sqrt{2}\vb{\Pi}_1\vb{VT}\ket{\gamma_0}$.
\begin{itemize}
\item
Perform $\vb{T}^\dagger \vb{V}^\dagger \ket{\delta_1} = \frac{1}{\sqrt{2}}\ket{\gamma_0} - \frac{1}{\sqrt{2}}\ket{\gamma_1}$
\item
Perform $(2\vb{\Delta}_0 - \idm_\reg{wx})(\frac{1}{\sqrt{2}}\ket{\gamma_0} - \frac{1}{\sqrt{2}}\ket{\gamma_1}) = \frac{1}{\sqrt{2}}\ket{\gamma_0} + \frac{1}{\sqrt{2}}\ket{\gamma_1}$
\item
Perform $ \vb{VT}(\frac{1}{\sqrt{2}}\ket{\gamma_0} + \frac{1}{\sqrt{2}}\ket{\gamma_1})= \green{\ket{\delta_0}}$. Go next step.
\end{itemize}
\end{itemize}
\item
Sim can finish the last round as the honest prover. 
\end{itemize}
\end{frame}

\begin{frame}
\frametitle{Extending to G3C---Idealized Com Model (1/3)}

\begin{itemize}
\item
The graph-3-coloring (G3C) problem is \NP-complete

\item
Start point: the G3C classical ZK proof from \cite{DBLP:conf/focs/GoldreichMW86}
\end{itemize}

Caveats:
\begin{itemize}
\item
$\Pr[\text{Guess correctly}] = \frac{1}{m}$, where $m=$ \# edges. 
\item
$\Pr[\text{Guess correctly}] ~\perp~ \ket{\psi}$?
\begin{itemize}
\item
Yes, if the 1st msg.\ is a perfect-hiding (PH) $\Com$
\begin{itemize}
\item
What about binding? --- Collapse-binding suffices \cite{DBLP:conf/eurocrypt/Unruh16}
\end{itemize}
\item 
No, if the 1st $\Com$ msg.\ is only statistically/computationally-hiding.
\item
We assume an ideal $\Com$ for simplicity: perfect-hiding and perfectly-binding
\item
Extends to comp.-hiding $\Com$ later 
\end{itemize} 
\end{itemize}
\end{frame}


\begin{frame}
\frametitle{Extending to G3C---Idealized Com Model (2/3)}

Key ingredients for the GI simulator:
\begin{itemize}
\item
Define an operator: $\vb{\Delta}^\dagger_0 \vb{T}^\dagger \vb{V}^\dagger \vb{\Pi}_0  \vb{V} \vb{T} \vb{\Delta}_0$ ($\eqqcolon \vb{M}$) 
\item
An technical Lemma \ref{lem:technical}: $\lambda=\frac{1}{2}$ $(\perp \ket{\psi})$
\item
Invoke Marriot-Watrous Lemma \ref{lem:marriot-watrous} with $\lambda=\frac{1}{2}$:
\begin{itemize}
\item Voil\`a \dSmiley[1.2]! We can get $\green{\ket{\delta_0}}$ within $\le 2$ steps
\end{itemize}
\end{itemize}
~\\
What will change for the G3C protocol?
\begin{itemize}
\item
$\vb{M}$ defined as before (w/ $\vb{T}$ and $\vb{V}$ modified in the natural way)
\item
Lemma \ref{lem:technical}: $\lambda=\frac{1}{m}$ $(\perp \ket{\psi})$
\item
Invoke Marriot-Watrous Lemma \ref{lem:marriot-watrous} with $\lambda=\frac{1}{m}$:
\begin{itemize}
\item
\dSadey[1.2]! no guarantee for $\green{\ket{\delta_0}}$ within $\le 2$ steps\end{itemize}
\item 
Solution: use the full power of Matrriot-Watrous analysis (next slide).
\end{itemize}

\end{frame}

\begin{frame}
\frametitle{Extending to G3C---Idealized Com Model (3/3)}
\blue{(draw the evolution diagram in the current setting)}
$$
\mqty{
\ket{\gamma_0} & ~~~~~~~~~~~~~~~ & \green{\ket{\delta_0}} & ~~~~~~~~~~~~~~~ & \ket{\gamma_0} & ~~~~~~~~~~~~~~~ & \green{\ket{\delta_0}} & ~~~~~~~~~~~~~~~ & \ket{\gamma_0} &\cdots\\ 
~ & ~ & ~ & ~ & ~ \\
~ & ~ & ~ & ~ & ~\\
~ & ~ & ~ & ~ & ~\\
~ & ~~~~~~~~~~~~~~~ & \ket{\delta_1} & ~~~~~~~~~~~~~~~ & \red{\ket{\gamma_1}} & ~~~~~~~~~~~~~~~ & \ket{\delta_1} &  ~~~~~~~~~~~~~~~ & \red{\ket{\gamma_0}} &\cdots
} 
$$
The main take-away:
\begin{itemize}
\item
{\footnotesize
$\mqty{
	\vb{U}\ket{\gamma_0} = \sqrt{\egval}\ket{\delta_0} + \sqrt{1 - \egval}\ket{\delta_1} & \vb{U}^\dagger\ket{\delta_0} = \sqrt{\egval}\ket{\gamma_0} + \sqrt{1-\egval}\ket{\gamma_1}\\
	\vb{U}\ket{\gamma_1} = \sqrt{1-\egval}\ket{\delta_0} - \sqrt{\egval}\ket{\delta_1}  & \vb{U}^\dagger\ket{\delta_1} = \sqrt{1-\egval}\ket{\gamma_0} - \sqrt{\egval}\ket{\gamma_1} 
}$}, where $\lambda = 1/m$.
\item
	Measure $\Set{\vb{\Pi}_0, \vb{\Pi}_1}$ at each $\ket{\delta}$, if results in $\ket{\delta_1}$:
	\begin{itemize}
	\item
	$\vb{U} (2 \vb{\Delta}_0 - \idm) \vb{U}^\dagger \ket{\delta_1} = 2\sqrt{p(1-p)}\ket{\delta_0} + (1-2p)\ket{\delta_1} $
	\end{itemize}
\item Measure $\Set{\vb{\Pi}_0, \vb{\Pi}_1}$. Go to $\ket{\delta_1}$ w.p.\ $(1-2p)$.
\item Prob.\ for continuous failure after $t$ iteration: $(1-p)(1-2p)^t$. Can be negl.\ by setting $t$ properly.
\end{itemize}
\end{frame}

\begin{frame}
\frametitle{The General Quantum Rewinding Lemma (Exact)}
{\small
\begin{LemmaBox}[label={lem:quantum-rewinding}]{Exact Quantum Rewinding \cite{DBLP:journals/siamcomp/Watrous09}}
$\vb{Q}$ is a QC works on ${\ket{\psi}}$ and with $\Pr[\mathsf{success}] = p$ $(\perp \ket{\psi})$ outputs $\green{\ket{\delta_0}}$. Then, for any $\epsilon>0$, there exists another QC $\vb{R}$ of size
$$O\left(\frac{\log(1/\epsilon)}{p(1-p)} \cdot \mathsf{size}(\vb{Q})\right)$$
such that for every input $\ket{\psi}$, the output $\rho$ of $\vb{R}$ satisfies $\green{\bra{\delta_0}} \rho \green{\ket{\delta_0}} \ge 1- \epsilon$.
\end{LemmaBox}}
\begin{itemize}
	\item $\green{\bra{\delta_0}} \rho \green{\ket{\delta_0}} =$  the squared {\em Fidelity} (i.e., $F^2(\rho, \green{\ketbra{\delta_0}})$)
	\begin{itemize}
	\item a metric for how close these two outputs are. (The closer to 1, the better) 
	\item relation to trace distance: $1 - F(\rho_1, \rho_2) \le \norm{\rho_1 - \rho_2}_\mathrm{tr} \le \sqrt{1 - F^2(\rho_1, \rho_2) }$
	\end{itemize}
	\item ``Exact'' refers to the face that $p\perp \ket{\psi}$.
\item
The $\frac{\log(1/\epsilon)}{p(1-p)}$: because we need a proper $t$ to achieve a negl.\ failure prob.
\item 
Only need poly-size for a negligible $\epsilon$.
\end{itemize}
\end{frame}

\begin{frame}
\frametitle{G3C ZK with Comp.-Hiding \Com}
\begin{itemize}
\item
(Sim's 1st msg.) $\cind$ (Prover's 1st msg.)
\item
($V^*$'s challenge $a$) $\not\perp$ (the 1st msg.)
\item
In Lemma \ref{lem:quantum-rewinding}, $\Pr[\mathsf{success}] = p(\ket{\psi})$. 
\begin{itemize}
\item
$p(\ket{\psi})$ jiggles within an negl.\ small interval.
\end{itemize}
\item
Need a version of Lemma \ref{lem:quantum-rewinding} allowing small perturbations
\end{itemize}
\end{frame}


\begin{frame}
\frametitle{The Version Allowing Small Perturbations}

\begin{LemmaBox}[label={lem:quantum-rewinding-perturbation}]{Quantum Rewinding with Small Perturbations {\cite[Sec. 4.2]{DBLP:journals/siamcomp/Watrous09}}}
Let $\vb{Q}$, $\ket{\psi}$, and $\green{\ket{\delta_0}}$ as before. But $\Pr[\mathsf{success}] = p(\ket{\psi})$ now depends on $\ket{\psi}$. Let $p_0, q \in (0,1)$ and $\epsilon \in (0,1/2)$ be real numbers such that\\[0.3em]
\centerline{$(1).~\abs{p(\psi) - q} < \epsilon~~~~~~~(2).~p_0 \le p(\psi)~~~~~~~(3).~p_0(1-p_0) \le q(1-q)$}\\[0.4em]
Then, for any $\epsilon>0$, there exists another QC $\vb{R}$ of size
$O\left(\frac{\log(1/\epsilon)}{\blue{p_0(1-p_0)}} \cdot \mathsf{size}(\vb{Q})\right)$
such that for every input $\ket{\psi}$, the output $\rho$ of $\vb{R}$ satisfies:\\[-0.5em]
$$ F^2(\rho, \green{\ketbra{\delta_0}}) = \green{\bra{\delta_0}} \rho \green{\ket{\delta_0}} \ge 1- \blue{16\epsilon\frac{\log^2(1/\epsilon)}{p^2_0(1-p_0)^2}}.$$
\end{LemmaBox}

Proof at a high-level:
\begin{itemize}
\item
Consider each eigen-space separately (next slide).
\item
For detailed calculation, see {\cite[Sec. 4.2]{DBLP:journals/siamcomp/Watrous09}}.
\end{itemize}
\end{frame}

\begin{frame}
\frametitle{Proof Sketch for Lemma \ref{lem:quantum-rewinding-perturbation}}
Proof Sketch:
\begin{itemize}
\item
In Lemma \ref{lem:technical},
$\ket{\gamma_0} = \ket{\psi}_\reg{w}\ket{0}_\reg{x}$ is no longer an evec.\ of $\vb{M}$
\begin{itemize}
\item 
The reason: $\ket{\psi}_\reg{w}$ is not an evec.\ of $\vb{Q}$
\end{itemize}
\item \blue{(mental exper.)}
Thus, decomp.\ $\ket{\psi}$ in the evecs $\Set{\ket{\psi_i}}_{i \in [\mathsf{dim}]}$ of $\vb{Q}$ 
\item \blue{(mental exper.)}
For each $i$, we obtain Lemmas \ref{lem:technical} and \ref{lem:marriot-watrous}
\item \blue{(mental exper.)}
In the Marriot-Watrous procedure, in each egein space:\\[0.2em]
\centerline{\small$\vb{VT}\ket{\psi_i}_\reg{w}\ket{0}_\reg{x} = \blue{\sqrt{p(\ket{\psi_i})}} \ket{\delta_0(\ket{\psi_i})} + \blue{\sqrt{1 - p(\ket{\psi_i})}}\ket{\delta_1(\ket{\psi_i})}$}
\item \blue{(mental exper.)}
Define a unitary $\vb{N}$ such that for all $i\in [\mathsf{dim}]$:\\[0.2em]
{\small
$\blue{\sqrt{p(\ket{\psi_i})}} \ket{\delta_0(\ket{\psi_i})} + \blue{\sqrt{1 - p(\ket{\psi_i})}}\ket{\delta_1(\ket{\psi_i})} \rightarrow \blue{\sqrt{q}} \ket{\delta_0(\ket{\psi_i})} + \blue{\sqrt{1 - q}}\ket{\delta_1(\ket{\psi_i})}$}
\item \blue{(mental exper.)}
Ready to apply the Exact Rewinding Lemma \ref{lem:quantum-rewinding} (w/ $p_0$ as we don't know $p$.) \blue{(Need $p_0(1-p_0) \le q(1-q)$.)}
\end{itemize}
In summary, this is a Sim w/ an imaginary operator $\vb{N}$, giving the same trace bound as in Lemma \ref{lem:quantum-rewinding}. But for the real Sim, there is no $\vb{N}$. 
\begin{itemize}
\item Doesn't matter. $\vb{N}$ only affects the trace bound negligibly.
\item By tedious-yet-elementary linear algebra (see {\cite[Sec. 4.2]{DBLP:journals/siamcomp/Watrous09}}). 
\end{itemize}

\end{frame}

\begin{frame}%[allowframebreaks]
\frametitle{References}
{\fontsize{10pt}{0pt}\selectfont
\bibliographystyle{alpha}
\bibliography{additionalRef}
}
\end{frame}


\end{document}