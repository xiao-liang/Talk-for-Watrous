\documentclass[aspectratio=1610, 12pt, xcolor={dvipsnames}]{beamer}
\usefonttheme{professionalfonts}
\usepackage{siunitx}
\usepackage{newtxtext}
\usepackage{bm}
% Possible aspect ration: 1610, 149, 54, 43 and 32.
\usepackage{Xiao-slides}

% tcolorbox macro
% Usage: \begin{mybox}{Box_title} contents \end{mybox}



% Macros:
% Case-insensitive, Alphabetical ordering
\newcommand{\Adv}{\ensuremath{\mathcal{A}}\xspace}
\newcommand{\AglB}[1]{\ensuremath{\langle #1 \rangle}}
\newcommand{\amount}{\ensuremath{\mathsf{amount}}\xspace}
\newcommand{\Aux}{\ensuremath{\mathsf{Aux}}\xspace}
\newcommand{\A}{\ensuremath{\mathsf{Alice}}\xspace}
\newcommand{\Bad}{\ensuremath{\mathsf{Bad}}\xspace}
\newcommand{\BBCom}{\ensuremath{\mathsf{BBCom}}\xspace}
\newcommand{\BBOWFP}{\ensuremath{\mathsf{BB}\text{-}\mathsf{OWFP}}\xspace}
\newcommand{\BBProve}{\ensuremath{\mathsf{BBProve}}\xspace}
\newcommand{\Bdv}{\ensuremath{\mathcal{B}}\xspace}
\newcommand{\bits}{\ensuremath{\{0,1\}}\xspace}
\newcommand{\blue}[1]{{\color{blue} #1}}
\newcommand{\BPP}{\ensuremath{\mathcal{BPP}}\xspace}
\newcommand{\ch}{\ensuremath{\mathsf{ch}}\xspace}
\newcommand{\cind}{\ensuremath{\stackrel{\text{c}}{\approx}}\xspace}
\newcommand{\Circuit}{\ensuremath{\mathsf{Cir}}\xspace}
\newcommand{\CKTVAL}{\ensuremath{\textsc{CktVal}}\xspace}
\newcommand{\cm}{\ensuremath{\mathsf{cm}}\xspace}
\newcommand{\com}{\ensuremath{\mathsf{com}}\xspace}
\newcommand{\Com}{\ensuremath{\mathsf{Com}}\xspace}
\newcommand{\ConPar}{\ensuremath{n}}
\newcommand{\CRHF}{\ensuremath{\mathsf{CRHF}}\xspace}
\newcommand{\cZK}{\ensuremath{c\mathcal{ZK}}\xspace}
\newcommand{\Dec}{\ensuremath{\mathsf{Dec}}\xspace}
\newcommand{\Dist}{\ensuremath{\mathcal{D}}\xspace}

\newcommand{\Domain}{\ensuremath{\mathsf{D}}\xspace}
\newcommand{\Dstng}{\ensuremath{\mathcal{D}}\xspace}

\newcommand{\Easy}{\ensuremath{\mathsf{Easy}}\xspace}
\newcommand{\ECC}{\ensuremath{\mathsf{ECC}}\xspace}
\newcommand{\Enc}{\ensuremath{\mathsf{Enc}}\xspace}
\newcommand{\Error}{\ensuremath{\delta}\xspace}
\newcommand{\Eval}{\ensuremath{\mathsf{Eval}}\xspace}
\newcommand{\Event}{\ensuremath{\mathsf{Event}}\xspace}
\newcommand{\Exec}{\ensuremath{\mathsf{Exec}}\xspace}
\newcommand{\ExtCom}{\ensuremath{\mathsf{ExtCom}}\xspace}
\newcommand{\Ext}{\ensuremath{\mathsf{Ext}}\xspace}

\newcommand{\Field}{\ensuremath{\mathbb{F}}\xspace}
\newcommand{\Func}{\ensuremath{\mathcal{F}}\xspace}

\newcommand{\given}[1][\big]{\:#1\vert\:}
\newcommand{\Good}{\ensuremath{\mathsf{Good}}\xspace}
\newcommand{\gray}[1]{{\color{Gray} #1}}
\newcommand{\green}[1]{{\color{ForestGreen} #1}}

\newcommand{\Ham}{\ensuremath{\Delta}\xspace}
\newcommand{\Hard}{\ensuremath{\mathsf{Hard}}\xspace}

\newcommand{\idind}{\ensuremath{\stackrel{\text{i.d.}}{=\mathrel{\mkern-3mu}=}}\xspace}
\newcommand{\IDEAL}{\ensuremath{\mathsf{IDEAL}}\xspace}
\newcommand{\Ind}{\ensuremath{\mathsf{Ind}}\xspace}
\newcommand{\Input}{\ensuremath{\mathsf{in}}\xspace}
\newcommand{\IP}{\ensuremath{\mathcal{IP}}\xspace}

\newcommand{\KGen}{\ensuremath{\mathsf{KGen}}\xspace}

\newcommand{\Lang}{\ensuremath{\mathcal{L}}\xspace}

\newcommand{\NCKTVAL}{\ensuremath{\textsc{NdCktVal}}\xspace}
\newcommand{\negl}{\ensuremath{\mathsf{negl}}\xspace}
\newcommand{\NoHard}{\ensuremath{\mathsf{NoHard}}\xspace}
\newcommand{\NP}{\ensuremath{\mathsf{NP}}\xspace}
\newcommand{\np}{\NP}

\newcommand{\om}[1]{{\noindent \color{magenta}OM: #1}}
\newcommand{\Ora}{\ensuremath{\mathcal{O}}\xspace}
\newcommand{\ora}{\ensuremath{\mathsf{O}}\xspace}
\newcommand{\Output}{\ensuremath{\mathsf{Out}}\xspace}
\newcommand{\OWFP}{\ensuremath{\mathsf{OWFP}}\xspace}
\newcommand{\OWF}{\ensuremath{\mathsf{OWF}}\xspace}
\newcommand{\owf}{\OWF}

\renewcommand{\P}{\ensuremath{\mathcal{P}}\xspace}
\newcommand{\Path}{\ensuremath{\mathsf{P}}\xspace}
\newcommand{\PBCom}{\ensuremath{\mathsf{PBCom}}\xspace}
\newcommand{\PHCom}{\ensuremath{\mathsf{PHCom}}\xspace}
% \newcommand{\pick}{\ensuremath{\xleftarrow{\$}}\xspace}
\newcommand{\pick}{\ensuremath{\leftarrow}\xspace}
\newcommand{\poly}{\ensuremath{\mathsf{poly}}\xspace}
\newcommand{\PPT}{\ensuremath{\mathrm{PPT}}\xspace}
\newcommand{\PRF}{\ensuremath{\mathsf{PRF}}\xspace}
\newcommand{\PRG}{\ensuremath{\mathsf{PRG}}\xspace}
\newcommand{\Prob}[2][]{\ensuremath{\Pr#1[ \text{#2} #1]}}
\newcommand{\Prot}{\ensuremath{\mathrm{\Pi}}\xspace}

\newcommand{\Range}{\ensuremath{\mathsf{R}}\xspace}
\newcommand{\REAL}{\ensuremath{\mathsf{REAL}}\xspace}
\newcommand{\Recon}{\ensuremath{\mathsf{Recon}}\xspace}
\newcommand{\red}[1]{{\color{red} #1}}
\newcommand{\Relation}{\ensuremath{\mathcal{R}}\xspace}
\newcommand{\RouPar}{\ensuremath{k}}

\newcommand{\SBCom}{\ensuremath{\mathsf{SBCom}}\xspace}
\newcommand{\SecPar}{\ensuremath{\lambda}\xspace}
\newcommand{\Seq}[1]{\ensuremath{\{#1\}}\xspace}
\newcommand{\Set}[1]{\ensuremath{\{#1\}}\xspace}
\newcommand{\Share}{\ensuremath{\mathsf{Share}}\xspace}
\newcommand{\SHCom}{\ensuremath{\mathsf{SHCom}}\xspace}
\newcommand{\Sim}{\ensuremath{\mathsf{Sim}}\xspace}
\newcommand{\sind}{\ensuremath{\stackrel{\text{s}}{\approx}}\xspace}
\newcommand{\ST}{\ensuremath{\mathsf{ST}}\xspace}

\newcommand{\Trans}{\ensuremath{\tau}\xspace}



\newcommand{\Valid}{\ensuremath{\mathsf{Valid}}\xspace}
\newcommand{\Vectbf}[1]{\ensuremath{\mathbf{#1}}\xspace}
\newcommand{\Vect}[1]{\ensuremath{\overline{#1}}\xspace}
\newcommand{\View}{\ensuremath{\mathsf{View}}\xspace}
\newcommand{\VSSCom}{\ensuremath{\mathsf{VSSCom}}\xspace}
\newcommand{\VSS}{\ensuremath{\mathsf{VSS}}\xspace}

\newcommand{\xiao}[1]{{\noindent \color{red}XIAO: #1}}
\newcommand{\xor}{\ensuremath{\oplus}\xspace}

\newcommand{\ZK}{\ensuremath{\mathsf{ZK}}\xspace}
\newcommand{\zk}{\ZK}
\newcommand{\ZKCnP}{\ensuremath{\mathsf{ZKCnP}}\xspace}
\newcommand{\ZKP}{\ensuremath{\mathsf{ZKP}}\xspace}
\newcommand{\zkp}{\ZKP}

%Information to be included in the title page:
% \title{The Watrous Post-Quantum Zero-Knowledge Proof}
% \subtitle{Presentation at the Max-Planck Institute Crypto Reading Group}
% \author{by\\~\\Xiao Liang}
% \institute{Stony Brook University \\and\\ Max Planck Institute (Security and Privacy)}
% \date{\today}
 
 
 
\begin{document}
 

% \frame{\titlepage}

\begin{frame}
\vspace{4em}
\centerline{\Large\beamerblue{The Watrous Post-Quantum Zero-Knowledge Proof}}
\vspace{3em}
\centerline{\beamerblue{A Crypto Reading Group Presentation}}
\vspace{1em}
\centerline{by}
\vspace{1em}
\centerline{Xiao Liang}
\vspace{2em}
\centerline{\small\textsc{Stony Brook University}}
\centerline{\small and}
\centerline{\small\textsc{Max Planck Institute (Security and Privacy)}}
\vspace{1em}
\centerline{\small\today}
\end{frame}


\begin{frame}
\frametitle{Post-Quantum ZK for NP}

The model:
\begin{itemize}
\item
Classical $P$ and $V$
\item
ZK system for \NP languages
\item
$V^*$ can be quantum. 
\begin{itemize}
\item
Modeled as a quantum polynomial-time (QPT) Turing machine.
\item
equivalently (and more preferred in quantum-computing literature), poly-size quantum circuits. 
\item 
Non-uniformity: $V^*$ has an auxiliary quantum state that depends only on the security para.\ $\SecPar$. More accurately,
$$V^* = \Set{\mathsf{QC}_\SecPar, \ket{\psi_\SecPar}}_{\SecPar\in \mathbb{N}}$$
\end{itemize}
\end{itemize}


\end{frame}


\begin{frame}
\frametitle{Post-Quantum (Black-Box) ZK Is Hard}

Why's {\bf rewinding} hard?
\begin{itemize}
\item
    information gain VS state disturbance
\item
 the no-cloning theorem
\end{itemize}
~\\
The major result in \cite{DBLP:conf/stoc/Watrous06}: a quantum rewinding lemma 
\end{frame}


\begin{frame}
\frametitle{Some Historical Notes}
Techniques inspired by Marriot-Watrous \cite{DBLP:conf/coco/MarriottW04} 
\begin{itemize}
\item error-gap amplification for QMA using only 1 witness state
\end{itemize}
~\\
First published at STOC'06 \cite{DBLP:conf/stoc/Watrous06}
\begin{itemize}
\item
explicit connection to \cite{DBLP:conf/coco/MarriottW04}
\item
we will see this version today 
\item my notation is consistent with this version
\end{itemize}
~\\
Then, SIAM Journal of Computing in 2009 \cite{DBLP:journals/siamcomp/Watrous09}
\begin{itemize}
\item
simplified, ad hoc proof
\item
hiding the connection with Marriot-Watrous
\item
so, we don't use this version today 
\end{itemize}
\end{frame}

\begin{frame}
        \frametitle{GMW ZK for Graph Isomorphism (GI)}

Some Remarks:
\begin{itemize}
\item
GI is not known to be \NP-complete.
\item
the 1st message of the GMW GI protocol is perfectly uniform.
\end{itemize}
~\\

{\bf Input for $P$:} statement $(G_0, G_1) \in \mathcal{G}_\SecPar \times\mathcal{G}_\SecPar$, witness $w = \sigma$ s.t.\ $\sigma(G_1) = G_0$

{\bf Input for $V$:}  $(G_0, G_1)$
\begin{enumerate}
\item 
$P$ samples $\pi \pick S_\SecPar$, sends $H = \pi(G_0)$
\item
$V$ sends $a \pick \bits$
\item
$P$ sends $\tau = \pi \circ \sigma^a$
\end{enumerate}
{\bf $V$'s decision:} accept iff $\tau(G_a) = H$
\\~

{\bf Classical Sim:} guess the bit $a$

\end{frame}



\begin{frame}
        \frametitle{Modeling in Quantum Way}


{\bf Model a Quantum $V^*$:} circuit family $\Set{\vb{V}_H}_{ H \in \mathcal{G}_\SecPar}$, auxiliary input $\ket{\psi}$
\begin{itemize}
	\item receives $H$ from P
	\item
	Perform $\vb{V}_H \ket{\psi}_\reg{w} \ket{0}_\reg{v} \ket{0}_\reg{a} = \alpha_0 \ket{\psi_0}_\reg{wv} \ket{0}_\reg{a} + \alpha_1 \ket{\psi_1}_\reg{wv} \ket{1}_\reg{a}$ 
	\begin{itemize}
	\item $\reg{v}$: work space
	\item $\reg{a}$: single-qubit register to store $V^*$'s challenge.
	\item Note that $\vb{V}_H$ operates on space $\reg{w}\tensor \reg{v} \tensor \reg{a}$
	\end{itemize}
	
\end{itemize}
\end{frame}

\begin{frame}
\frametitle{Modeling in Quantum Way}
View the protocol through a quantum lens: 
\begin{itemize}
	\item
	The full space $\reg{w} \tensor \reg{x}$, where $\reg{x} = \reg{v} \tensor \reg{a} \tensor \reg{y} \tensor \reg{b} \tensor \reg{z}$
\item
$P$ performs $$\vb{T} \ket{0}_\reg{YBZ} = \frac{1}{\sqrt{2\SecPar!}} \sum_{b\in \bits} \sum_{\pi \in S_\SecPar} \ket{\pi(G_b)}_\reg{y} \ket{b}_\reg{B} \ket{\pi}_\reg{z}$$
\item
$V$ apply $\vb{V} = \sum_{H \in \mathcal{G}} \vb{V}_H \tensor \ketbra{H}_\reg{y} \tensor \idm_\reg{bz}$ on the full space $\reg{w}\tensor\reg{x}$
\begin{itemize}
	\item recall that $\vb{V}_H$ operates on $\ket{\psi}_\reg{w} \ket{0}_\reg{v} \ket{0}_\reg{a}$
\item 
corresponding to the exec.\ in super-position
\item
Output format:
$$ \alpha_{00}\ket{\psi_{00}}\ket{00}_\reg{ab} +  \alpha_{01}\ket{\psi_{01}}\ket{01}_\reg{ab} + \alpha_{10}\ket{\psi_{10}}\ket{10}_\reg{ab} +\alpha_{11}\ket{\psi_{11}}\ket{11}_\reg{ab}$$
\end{itemize}
\end{itemize}
In summary, the protocol up to step 2 is:
$$\underbrace{\vb{V}\vb{T}}_{\text{on}~\reg{w}\tensor \reg{x}} (\ket{\psi}_\reg{w} \ket{0}_{\reg{x}= \reg{VAYBZ}}) ~~\Iff~~ \underbrace{\vb{V}\vb{T}(\idm_\reg{w} \tensor \ket{0}_\reg{x})}_{\text{only on}~\reg{w}} \ket{\psi}$$
\end{frame}

\begin{frame}
\frametitle{Measuring the Guess}

Define a binary-outcome measurement on the full space $\reg{w}\tensor\reg{x}$: 
\begin{itemize}
\item $\vb{\Pi}_0 = \ketbra{0}_\reg{ab}, ~~~ \vb{\Pi}_1 \coloneqq \idm_\reg{ab} - \ketbra{0}_\reg{ab} = \ketbra{1}_\reg{ab}$
\item 
work on the full space $\reg{w}\tensor\reg{x}$. Just tensor identities on registers other than $\reg{ab}$
\end{itemize}
~\\


Performing $\Set{\vb{\Pi}_0, \vb{\Pi}_1}$ on $\vb{V}\vb{T}\ket{\psi}_\reg{w}\ket{0}_\reg{x}$ boils down to:
\begin{itemize}
\item the probability for outcome $0$ is 
$\Tr\big(\bra{\psi}\vb{Q} \ket{\psi}\big)$
\item the probability for outcome $1$ is 
$\Tr\big(\bra{\psi}( \idm_\reg{w}- \vb{Q} )\ket{\psi}\big)$
\end{itemize}
where $\vb{Q}	 = (\idm_\reg{w} \tensor \bra{0}_\reg{x}) \vb{T}^\dagger \vb{V}^\dagger \vb{\Pi}_0 \vb{T} \vb{V} (\idm_\reg{w} \tensor \ket{0}_\reg{x})$.
\\~

Two important facts:
\begin{itemize}
\item $\Set{\vb{Q}, \idm_\reg{w}- \vb{Q}}$ form a POVM
\item
$\Tr\big(\bra{\psi}\vb{Q} \ket{\psi}\big) = \Tr\big(\bra{\psi}( \idm_\reg{w}- \vb{Q} )\ket{\psi}\big) = \frac{1}{2}$, \blue{independent of $\ket{\psi}$}. (Cuz 1st msg.\ of GI prot.\ is perfectly uniform.)
\end{itemize}
$\Rightarrow ~~ \vb{Q}  = \idm_\reg{w}- \vb{Q} = \frac{1}{2}\idm_\reg{w}$
\end{frame}



\begin{frame}
\frametitle{An Important Lemma}
Let $\vb{\Delta}_0 \coloneqq \idm_\reg{w} \tensor \ketbra{0}_\reg{x}$.
\begin{itemize}
\item 
$\vb{\Delta}_0$ projects register $\reg{x}$ to all-$0$ qubits. 
\item
$\vb{\Delta}_0 = \vb{\Delta}^\dagger_0$ 
\item
$\vb{\Delta}_1 \coloneqq \idm_\reg{wx}  -  \vb{\Delta}_0$. The $\Set{\vb{\Delta}_0, \vb{\Delta}_1}$ form a POVM.
\end{itemize}
\begin{LemmaBox}[label={lem:technical}]{}
 For all $\ket{\psi}\in \mathcal{H}(\reg{w})$, $\ket{\gamma_0} = \ket{\psi}_\reg{w}\ket{0}_\reg{x}$ is an eigenvector of $\vb{\Delta}^\dagger_0 \vb{T}^\dagger \vb{V}^\dagger \vb{\Pi}_0 \vb{T} \vb{V} \vb{\Delta}_0$ with corresponding eigenvalue $\lambda = 1/2$.
\end{LemmaBox}
{\bf Proof.} Recall $\vb{Q}	 = (\idm_\reg{w} \tensor \bra{0}_\reg{x}) \vb{T}^\dagger \vb{V}^\dagger \vb{\Pi}_0 \vb{T} \vb{V} (\idm_\reg{w} \tensor \ket{0}_\reg{x}) = \frac{1}{2}\idm_\reg{w}$.
\begin{align*}
\Rightarrow ~~ & 
\vb{\Delta}^\dagger_0 \vb{T}^\dagger \vb{V}^\dagger \vb{\Pi}_0 \vb{T} \vb{V} \vb{\Delta}_0 = (\idm_\reg{w}\tensor\ket{0}_\reg{x}) \vb{Q} (\idm_\reg{w}\tensor\bra{0}_\reg{x}) = \frac{1}{2} \idm_\reg{w} \tensor \ketbra{0}_\reg{x} \\
\Rightarrow ~~ &  \forall \ket{\psi}, \vb{\Delta}^\dagger_0 \vb{T}^\dagger \vb{V}^\dagger \vb{\Pi}_0 \vb{T} \vb{V} \vb{\Delta}_0 \underbrace{\ket{\psi}_\reg{w} \ket{0}_\reg{x}}_{\ket{\gamma_0}} = \big(\frac{1}{2} \idm_\reg{w} \tensor \ketbra{0}_\reg{x}\big) \underbrace{\ket{\psi}_\reg{w} \ket{0}_\reg{x}}_{\ket{\gamma_0}} = \frac{1}{2}\underbrace{\ket{\psi}_\reg{w} \ket{0}_\reg{x}}_{\ket{\gamma_0}}
\end{align*}
\end{frame}



\begin{frame}
\frametitle{Marriot-Watrous Lemma}
{\fontsize{9pt}{0pt}\selectfont
\begin{LemmaBox}[label={lem:marriot-watrous}]{Marriot-Watrous \cite{DBLP:conf/coco/MarriottW04}}
Given unitary $\vb{U}$, proj.\ mnt.\ $\Set{\vb{\Pi}_0, \vb{\Pi}_1}$ and $\Set{\vb{\Delta}_0, \vb{\Delta}_1}$. 
Assume $\ket{\gamma_0}$ is an evec.\ of $\vb{\Delta}_0\vb{U}^\dagger\vb{\Pi}_0\vb{U}\vb{\Delta}_0$ with eval.\ $\egval$. Define 
$$\ket{\delta_0} \coloneqq \frac{\vb{\Pi}_0\vb{U}\ket{\gamma_0}}{\sqrt{\egval}}, ~~ \ket{\delta_1} \coloneqq \frac{\vb{\Pi}_0\vb{U}\ket{\gamma_0}}{\sqrt{1-\egval}},~~ \ket{\gamma_1} \coloneqq \frac{\vb{\Delta}_1\vb{U}^\dagger\ket{\delta_0}}{\sqrt{1-\egval}}.$$
Then, $\braket{\gamma_0}{\gamma_1} = \braket{\delta_0}{\delta_1} = 0$ and 
\begin{align*}
\vb{U}\ket{\gamma_0} = \sqrt{\egval}\ket{\delta_0} + \sqrt{1 - \egval}\ket{\delta_1}~~~~~~~~~~~~ & \vb{U}^\dagger\ket{\delta_0} = \sqrt{\egval}\ket{\gamma_0} + \sqrt{1-\egval}\ket{\gamma_1} \\
\vb{U}\ket{\gamma_1} = \sqrt{1-\egval}\ket{\delta_0} - \sqrt{\egval}\ket{\delta_1} ~~~~~~~~~~~~ & \vb{U}^\dagger\ket{\delta_1} = \sqrt{1-\egval}\ket{\gamma_0} - \sqrt{\egval}\ket{\gamma_1} 
\end{align*}

\end{LemmaBox}}
\blue{(draw the diagram)}
$$
\mqty{
\ket{\gamma_0} & ~~~~~~~~~~~~~~~ & \green{\ket{\delta_0}} & ~~~~~~~~~~~~~~~ & \ket{\gamma_0} & ~~~~~~~~~~~~~~~ & \green{\ket{\delta_0}} & ~~~~~~~~~~~~~~~ & \ket{\gamma_0} &\cdots\\ 
~ & ~ & ~ & ~ & ~ \\
~ & ~ & ~ & ~ & ~\\
~ & ~ & ~ & ~ & ~\\
~ & ~~~~~~~~~~~~~~~ & \ket{\delta_1} & ~~~~~~~~~~~~~~~ & \red{\ket{\gamma_1}} & ~~~~~~~~~~~~~~~ & \ket{\delta_1} &  ~~~~~~~~~~~~~~~ & \red{\ket{\gamma_0}} &\cdots
} 
$$
\end{frame}


\begin{frame}
\frametitle{In Our Setting: Marriot-Watrous $+$ Post-Mnt Selection}
In our setting, we have $\vb{U} = \vb{V}\vb{T}$, $\lambda = 1/2$, and $\ket{\gamma_0} = \ket{\psi}_\reg{w} \ket{0}_\reg{x}$\\[0.5em]

Lemma \ref{lem:marriot-watrous}~ $\Rightarrow ~ \ket{\gamma_0} = \frac{1}{\sqrt{2}}\ket{\delta_0} + \frac{1}{\sqrt{2}}\ket{\delta_1}$, and the following:
{\fontsize{10pt}{0pt}\selectfont$$
\ket{\delta_0} = \sqrt{2}\vb{\Pi}_0 \vb{V} \vb{T}\ket{\gamma_0}, ~~~~\vb{T}^\dagger \vb{V}^\dagger \ket{\delta_1} = \frac{1}{\sqrt{2}}\ket{\gamma_0} - \frac{1}{\sqrt{2}}\ket{\gamma_1}, ~~~~\vb{VT} (\frac{1}{\sqrt{2}}\ket{\gamma_0} + \frac{1}{\sqrt{2}}\ket{\gamma_1}) = \ket{\delta_0}
$$}

% $~~\ket{\delta_1} = \sqrt{2}\vb{\Pi}_1 \vb{V} \vb{T}\ket{\gamma_0}, ~~ \ket{\gamma_1} = \sqrt{2}\vb{\Delta}_1 \vb{T}^\dagger \vb{V}^\dagger\ket{\gamma_0}$
Starting with $\ket{\gamma_0} \rightarrow \vb{VT} \ket{\gamma_0} \rightarrow$ measurement $\Set{\vb{\Pi}_0, \vb{\Pi}_1}$:
\begin{itemize}
\item
w.p.\ $1/2$, it is $\green{\ket{\delta_0}}$ --- we are done!
\item
w.p.\ $1/2$, it is $\ket{\delta_1}$
\begin{itemize}
\item Key observation: $\vb{T}^\dagger \vb{V}^\dagger \ket{\delta_1} = \frac{1}{\sqrt{2}}\ket{\gamma_0} - \frac{1}{\sqrt{2}}\ket{\gamma_1}$
\item \red{If we can flip the phase of the 2nd term} $\Rightarrow ~~\frac{1}{\sqrt{2}}\ket{\gamma_0} + \frac{1}{\sqrt{2}}\ket{\gamma_1}$.
\item
Then, simply do $\vb{VT}(\frac{1}{\sqrt{2}}\ket{\gamma_0} + \frac{1}{\sqrt{2}}\ket{\gamma_1}) = \green{\ket{\delta_0}}$
\end{itemize}
\end{itemize}
\red{Yes, we can!} (next slide)
\end{frame}

\begin{frame}
\frametitle{Phase Flip for the 2nd Term}

We want: $\frac{1}{\sqrt{2}}\ket{\gamma_0} - \frac{1}{\sqrt{2}}\ket{\gamma_1} ~~~\rightarrow~~~ \frac{1}{\sqrt{2}}\ket{\gamma_0} + \frac{1}{\sqrt{2}}\ket{\gamma_1}$ \\[0.5em]

Recall the following 
\begin{itemize}
\item
$\ket{\gamma_0} = \ket{\psi}_\reg{w} \ket{0}_\reg{x}$ and $\vb{\Delta}_0 = \idm_\reg{w}\tensor\ketbra{0}_\reg{x}$
\item
$\Rightarrow ~~ \vb{\Delta}_0 \ket{\gamma_0} = \ket{\gamma_0}$
\item
Lemma \ref{lem:marriot-watrous} says $\ket{\gamma_1} = \sqrt{2}\vb{\Delta}_1 \vb{T}^\dagger \vb{V}^\dagger \ket{\delta_0} ~~\Rightarrow ~~ \vb{\Delta}_0 \ket{\gamma_1} = 0$
\end{itemize}

Therefore, it is not hard to come up with the following idea:
\begin{align*}
\underbrace{(2\vb{\Delta}_0 - \idm_\reg{wx})}_{=\vb{\Delta}_0 - \vb{\Delta}_1} (\frac{1}{\sqrt{2}}\ket{\gamma_0} - \frac{1}{\sqrt{2}}\ket{\gamma_1})
&= \frac{2}{\sqrt{2}}\vb{\Delta}_0 \ket{\gamma_0} - \frac{2}{\sqrt{2}}\vb{\Delta}_0 \ket{\gamma_1} - \frac{1}{\sqrt{2}}\ket{\gamma_0} + \frac{1}{\sqrt{2}}\ket{\gamma_1} \\
&= \frac{2}{\sqrt{2}}\ket{\gamma_0} - 0 - \frac{1}{\sqrt{2}}\ket{\gamma_0} + \frac{1}{\sqrt{2}}\ket{\gamma_1} \\
&= \frac{1}{\sqrt{2}}\ket{\gamma_0} + \frac{1}{\sqrt{2}}\ket{\gamma_1}
\end{align*}
\end{frame}





\begin{frame}
\frametitle{Summarizing the Watrous Simulator}
\begin{itemize}
\item
Start with $\ket{\gamma_0}_\reg{xw}=\ket{\psi}_\reg{x} \ket{0}_\reg{w}$
\item
Perform $\vb{VT}\ket{\gamma_0}_\reg{xw}$
\item
Perform measurement $\Set{\vb{\Pi}_0, \vb{\Pi}_1}$
\begin{itemize}
\item If outcome is 0 --- guessed correctly (in $\green{\ket{\delta_0}}$). Go next step.
\item Otherwise, we are in $\ket{\delta_1} = \sqrt{2}\vb{\Pi}_1\vb{VT}\ket{\gamma_0}$.
\begin{itemize}
\item
Perform $\vb{T}^\dagger \vb{V}^\dagger \ket{\delta_1} = \frac{1}{\sqrt{2}}\ket{\gamma_0} - \frac{1}{\sqrt{2}}\ket{\gamma_1}$
\item
Perform $(2\vb{\Delta}_0 - \idm_\reg{wx})(\frac{1}{\sqrt{2}}\ket{\gamma_0} - \frac{1}{\sqrt{2}}\ket{\gamma_1}) = \frac{1}{\sqrt{2}}\ket{\gamma_0} + \frac{1}{\sqrt{2}}\ket{\gamma_1}$
\item
Perform $ \vb{VT}(\frac{1}{\sqrt{2}}\ket{\gamma_0} + \frac{1}{\sqrt{2}}\ket{\gamma_1})= \green{\ket{\delta_0}}$. Go next step.
\end{itemize}
\end{itemize}
\item
Sim can finish the last round as the honest prover. 
\end{itemize}
\end{frame}


\begin{frame}%[allowframebreaks]
        \frametitle{References}
\bibliographystyle{alpha}
\bibliography{additionalRef}
\end{frame}


\end{document}