\documentclass[aspectratio=1610, 12pt]{beamer}
\usefonttheme{professionalfonts}
\usepackage{siunitx}
\usepackage{newtxtext}
\usepackage{bm}
% Possible aspect ration: 1610, 149, 54, 43 and 32.
\usepackage{Xiao-slides}
%Information to be included in the title page:
\title{The Watrous ZK Proof}

\subtitle{}
\author{}
\institute{}
\date{\today}
 
 
 
\begin{document}
 

\frame{\titlepage}




\begin{frame}
\frametitle{Post-Quantum ZK for NP}

The model:
\begin{itemize}
\item
Classical $P$ and $V$
\item
ZK system for \NP languages
\item
$V^*$ can be quantum. 
\begin{itemize}
\item
Modeled as a quantum polynomial-time (QPT) Turing machine.
\item
equivalently (and more preferred in QC literature), poly-size quantum circuits. 
\end{itemize}
\end{itemize}
\end{frame}


\begin{frame}
\frametitle{Post-Quantum (Black-Box) ZK Is Hard}

Why's {\bf rewinding} hard?
\begin{itemize}
\item
    information gain VS state disturbance
\item
 the no-cloning theorem
\end{itemize}
~\\
This work --- a quantum rewinding lemma 
\end{frame}


\begin{frame}
\frametitle{Some Historical Notes}
Techniques inspired by Marriot-Watrous \cite{DBLP:conf/coco/MarriottW04} 
\begin{itemize}
\item error-gap amplification for QMA using only 1 witness state
\end{itemize}
~\\
First published at STOC'06 \cite{DBLP:conf/stoc/Watrous06}
\begin{itemize}
\item
explicit connection to \cite{DBLP:conf/coco/MarriottW04}
\item
we will see this version today
\end{itemize}
~\\
Then, SIAM Journal of Computing in 2009 \cite{DBLP:journals/siamcomp/Watrous09}
\begin{itemize}
\item
simplified, ad hoc proof
\item
hiding the connection with Marriot-Watrous
\item
so, we don't use this version today
\end{itemize}
\end{frame}



\begin{frame}%[allowframebreaks]
        \frametitle{References}
\bibliographystyle{alpha}
\bibliography{additionalRef}
\end{frame}


\end{document}